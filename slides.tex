% Created 2016-10-28 Fri 14:51
\documentclass[presentation]{beamer}
\usepackage[utf8]{inputenc}
\usepackage[T1]{fontenc}
\usepackage{fixltx2e}
\usepackage{graphicx}
\usepackage{grffile}
\usepackage{longtable}
\usepackage{wrapfig}
\usepackage{rotating}
\usepackage[normalem]{ulem}
\usepackage{amsmath}
\usepackage{textcomp}
\usepackage{amssymb}
\usepackage{capt-of}
\usepackage{hyperref}
\usepackage{minted}
\usepackage{bussproofs}
\mode<beamer>{\usetheme{Madrid}}
\usepackage{upquote}
\usetheme{default}
\author{Phil Scott}
\date{27 October 2016}
\title{Write your own Theorem Prover}
\hypersetup{
 pdfauthor={Phil Scott},
 pdftitle={Write your own Theorem Prover},
 pdfkeywords={},
 pdfsubject={},
 pdfcreator={Emacs 25.1.1 (Org mode 8.3.6)}, 
 pdflang={English}}
\begin{document}

\maketitle

\begin{frame}[label={sec:orgheadline1}]{Introduction}
We'll work through a \emph{toy} LCF style theorem prover for classical propositional logic. We
will:

\begin{itemize}
\item review the LCF architecture
\item choose a logic
\item write the kernel
\item derive basic theorems/inference rules
\item build basic proof tools
\item write a decision procedure
\end{itemize}
\end{frame}

\begin{frame}[label={sec:orgheadline2}]{What is LCF?}
\begin{itemize}
\item A design style for theorem provers.
\item Follows the basic design of \emph{Logic of Computable Functions} (Milner, 1972).
\item Examples: HOL, HOL Light, Isabelle, Coq.
\item Syntax given by a data type whose values are logical terms.
\item There is an abstract type whose values are logical theorems.
\item Basic inference rules are functions on the abstract theorem type.
\item Derived rules are functions which call basic inference rules.
\end{itemize}
\end{frame}

\begin{frame}[label={sec:orgheadline3}]{What is Classical Propositional Logic (informally)}
\begin{itemize}
\item Syntax:
\begin{itemize}
\item Variables \(P, Q, \ldots, R\) and connectives \(\neg, \vee, \wedge, \rightarrow, \leftrightarrow\)
\item Terms/formulas: \(P\), \(\neg P\), \(P \vee Q\), \(P \wedge Q\), \(P \rightarrow Q\), 
\(P \leftrightarrow Q\)
\end{itemize}

\item Semantics
\begin{itemize}
\item Truth values \(\top\) and \(\bot\) assigned to variables
\item Connectives evaluate like ``truth-functions"; e.g. \(\top \vee \bot = \top\)
\item Theorems are terms which always evaluate to \(\top\) (tautologies)
\end{itemize}

\item Proof
Theorems can be found by truth-table checks, DPLL proof-search, or by applying
\emph{rules of inference to axioms}.
\end{itemize}
\end{frame}

\begin{frame}[label={sec:orgheadline4}]{An inference system for propositional logic}
\begin{itemize}
\item Given an alphabet \(\alpha\), a term is one of
\begin{itemize}
\item a variable \(v \in \alpha\)
\item an implication \(\psi \rightarrow \phi\) for formulas \(\phi\) and \(\psi\) (we take
\(\rightarrow\) to be right-associative)
\item a negation \(\neg\phi\) for some formula \(\phi\)
\end{itemize}

\item A theorem is one of
\begin{description}
\item[{Axiom 1}] \(\phi \rightarrow \psi \rightarrow \phi\) for terms \(\phi\) and \(\psi\)
\item[{Axiom 2}] \((\phi \rightarrow \psi \rightarrow \chi) \rightarrow (\phi
       \rightarrow \psi) \rightarrow \phi \rightarrow \chi\)
       for terms \(\phi\), \(\psi\) and \(\chi\)
\item[{Axiom 3}] \((\neg \phi \rightarrow \neg \psi) \rightarrow \psi \rightarrow \phi\)
       for terms \(\phi\) and \(\psi\)
\item[{Modus Ponens}] a term \(\psi\) whenever \(\phi\) and \(\phi \rightarrow \psi\) are
theorems
\end{description}
\end{itemize}
\end{frame}

\begin{frame}[fragile,label={sec:orgheadline5}]{The Kernel (syntax)}
 \begin{block}{Formally}
Given an alphabet \(\alpha\), a term is one of
\begin{itemize}
\item a variable \(v \in \alpha\)
\item an implication \(\psi \rightarrow \phi\) for formulas \(\phi\) and \(\psi\) (we take \(\rightarrow\) to be
right-associative)
\item a negation \(\neg\phi\) for some formula \(\phi\)
\end{itemize}
\end{block}

\begin{block}{\emph{Really} Formally}
\begin{minted}[]{haskell}
infixr 1 :=>:

data Term a = Var a
            | Term a :=>: Term a
            | Not (Term a)
            deriving Eq
\end{minted}
\end{block}
\end{frame}

\begin{frame}[fragile,label={sec:orgheadline6}]{Theorems}
 \small
\begin{minted}[]{haskell}
axiom1 :: a -> a -> Theorem a
axiom1 p q = Theorem (p :=>: q :=>: p)

axiom2 :: a -> a -> a -> Theorem a
axiom2 p q r =
  Theorem ((p :=>: q :=>: r) :=>: (p :=>: q) :=>: (p :=>: r))

axiom3 :: a -> a -> Theorem a
axiom3 p q = Theorem ((Not p :=>: Not q) :=>: q :=>: p)

mp :: Eq a => Theorem a -> Theorem a -> Theorem a
mp (Theorem (p :=>: q)) (Theorem p') | p == p' = Theorem q
\end{minted}
\end{frame}

\begin{frame}[fragile,label={sec:orgheadline7}]{Securing the Kernel}
 \begin{minted}[]{haskell}
module Proposition (Theorem, Term(..), termOfTheorem,
                    axiom1, axiom2, axiom3, mp ) where
\end{minted}

The \texttt{Theorem} type does not have any publicly visible constructors. The only way to
obtain values of \texttt{Theorem} type is to use the axioms and inference rule.
\end{frame}

\begin{frame}[label={sec:orgheadline8}]{First (meta) theorem}
\begin{theorem}
For any term \(P\), \(P \rightarrow P\) is a theorem.
\end{theorem}

\begin{proof}
Take \(\phi\) and \(\chi\) to be \(P\) and \(\psi\) to be \(P \rightarrow P\) in Axioms 1 and
2 to get:

\begin{enumerate}
\item \(P \rightarrow (P \rightarrow P) \rightarrow P\)
\item \((P \rightarrow (P \rightarrow P) \rightarrow P) \rightarrow (P \rightarrow P
     \rightarrow P) \rightarrow (P \rightarrow P)\)

Apply modus ponens to 1 and 2 to get:

\item \((P \rightarrow P \rightarrow P) \rightarrow P \rightarrow P\)

Use Axiom 1 with \(/phi\) and \(/psi\) to be \(P\) to get:

\item \((P \rightarrow P \rightarrow P)\)
\end{enumerate}

Apply modus ponens to 3 and 4.
\end{proof}
\end{frame}

\begin{frame}[fragile,label={sec:orgheadline9}]{First meta theorem formally}
 \begin{block}{Metaproof}
\begin{minted}[]{haskell}
theorem :: Eq a => Term a -> Theorem a
theorem p =
  let step1 = axiom1 p (p :=>: p)
      step2 = axiom2 p (p :=>: p) p
      step3 = mp step2 step1
      step4 = axiom1 p p
  in mp step3 step4
\end{minted}
\end{block}

\begin{block}{Example}
\begin{minted}[]{text}
> theorem (Var "P")
Theorem (Var "P" :=>: Var "P")
\end{minted}
>
\end{block}
\end{frame}
\begin{frame}[fragile,label={sec:orgheadline10}]{Issues}
 \begin{itemize}
\item How many axioms are there?

\begin{minted}[]{haskell}
axiom1 :: a -> a -> Theorem a
\end{minted}

\item How many theorems did we just prove?

\begin{minted}[]{haskell}
theorem :: Eq a => Term a -> Theorem a
\end{minted}

\item Why could this be a problem for doing formal proofs?
\end{itemize}
\end{frame}

\begin{frame}[fragile,label={sec:orgheadline11}]{A more(?) efficient axiomatisation}
 \small
\begin{minted}[]{haskell}
(p,q,r) = (Var 'p', Var 'q', Var 'r')
axiom1 :: Theorem Char
axiom1 = Theorem (p :=>: q :=>: p)

axiom2 :: Theorem Char
axiom2 = Theorem ((p :=>: q :=>: r)
                   :=>: (p :=>: q) :=>: (p :=>: r))

axiom3 :: Theorem Char
axiom3 = Theorem ((Not p :=>: Not q) :=>: q :=>: p)

instTerm :: (a -> Term b) -> Term a -> Term b
instTerm f (Var x)    = f x
instTerm f (Not t)    = Not (instTerm f t)
instTerm f (a :=>: c) = instTerm f a :=>: instTerm f c

inst :: (a -> Term b) -> Theorem a -> Theorem b
inst f (Theorem x) = Theorem (instTerm f x)
\end{minted}
\end{frame}

\begin{frame}[fragile,label={sec:orgheadline12}]{Metaproof again}
 \begin{block}{}
\begin{minted}[]{haskell}
truthThm =
  let inst1 = inst (\v -> if v == 'q' then p :=>: p else p)
      step1 = inst1 axiom1
      step2 = inst1 axiom2
      step3 = mp step2 step1
      step4 = inst (const p) axiom1
  in mp step3 step4
\end{minted}
\end{block}

\begin{block}{}
\begin{minted}[]{text}
> theorem
Theorem (Var 'P' :=>: Var 'P')
\end{minted}
\end{block}
\end{frame}

\begin{frame}[fragile,label={sec:orgheadline13}]{Derived syntax}
 \small
\begin{minted}[]{haskell}
infixl 4 \/
infixl 5 /\

-- | Syntax sugar for disjunction
(\/) :: Term a -> Term a -> Term a
p \/ q = Not p :=>: q

-- | Syntax sugar for conjunction
(/\) :: Term a -> Term a -> Term a
p /\ q  = Not (p :=>: Not q)

-- | Syntax sugar for truth
truth :: Term Char
truth = p :=>: p

-- | Syntax sugar for false
false :: Term Char
false = Not truth
\end{minted}
\end{frame}

\begin{frame}[label={sec:orgheadline14}]{A proof tool: the deduction [meta]-theorem}
\begin{block}{}
Why did we need five steps to prove \(P \rightarrow P\). Can't we just use
conditional proof?

\begin{enumerate}
\item Assume \(P\).
\item Have \(P\).
\end{enumerate}

Hence, \(P \rightarrow P\).
\end{block}

\begin{block}{Deduction Theorem}
From \(\{P\} \cup \Gamma \vdash Q\), we can derive \(\Gamma \vdash P \rightarrow Q\).
\end{block}

\begin{block}{}
But Our axiom system says nothing about assumptions!
\end{block}
\end{frame}

\begin{frame}[fragile,label={sec:orgheadline15}]{A DSL for proof trees with assumptions}
 \begin{block}{Syntax}
\begin{minted}[]{haskell}
data Proof a = Assume (Term a)
             | UseTheorem (Theorem a)
             | MP (Proof a) (Proof a)
           deriving Eq
\end{minted}
\end{block}

\begin{block}{Semantics}
\begin{minted}[,mathescape=t]{haskell}
-- Convert a proof tree to the form $\Gamma \vdash P$
sequent :: (Eq a, Show a) => Proof a -> ([Term a], Term a)
sequent (Assume a)   = ([a], a)
sequent (UseTheorem t) = ([], termOfTheorem t)
sequent (MP pr pr')    =
  let (asms,  p :=>: q)   = sequent pr
      (asms', _) = sequent pr' in
  (nub (asms ++ asms'), q)
\end{minted}
\end{block}
\end{frame}

\begin{frame}[fragile,label={sec:orgheadline16}]{A DSL for proof trees with assumptions}
 \begin{block}{Semantics}
\begin{minted}[,mathescape=t]{haskell}
-- Send $\{P\} \cup \Gamma \vdash Q$ to $\Gamma \vdash P \rightarrow Q$
discharge :: (Ord a, Show a) => Term a -> Proof a -> Proof a

-- Push a proof through the kernel
verify :: Proof a -> Theorem a
\end{minted}
\end{block}

\begin{block}{}
The implementation of `discharge` follows the proof of the deduction theorem!
\end{block}
\end{frame}

\begin{frame}[fragile,label={sec:orgheadline17}]{Example with DSL}
 \begin{block}{We want:}
\small
\begin{minted}[,mathescape=t]{haskell}
inst2 :: Term a -> Term a -> Theorem a -> Theorem a

-- $\vdash \neg P \rightarrow P \rightarrow \bot$
lemma1 =
  let step1 = Assume (Not p)
      step2 = UseTheorem (inst2 (Not p) (Not (false P)) axiom1)
      step3 = MP step2 step1
      step4 = UseTheorem (inst2 (false P) p axiom3)
      step5 = MP step4 step3
  in verify step5
\end{minted}

\begin{minted}[]{text}
> lemma1
Theorem (Not (Var 'P') :=>: Var 'P'
           :=>: Not (Var 'P' :=>: Var 'P'))
\end{minted}
\end{block}
\end{frame}

\begin{frame}[fragile,label={sec:orgheadline18}]{Embedding Sequent Calculus}
 \begin{block}{Assumption carrying proofs}
\begin{itemize}
\item We'd like to work with proofs of the form \(\Gamma \vdash P\) without needing a
DSL and a separate verification step.
\item We can identify a sequent \({P_1, P_2, \ldots, P_n} \vdash P\) with the implication
\(P_1 \rightarrow P_1 \rightarrow \cdots \rightarrow P_n \rightarrow P\)
\item We just need to keep track of \(n\):
\end{itemize}

\begin{minted}[]{haskell}
data Sequent a = Sequent Int (Theorem a)
\end{minted}
\end{block}
\end{frame}

\begin{frame}[label={sec:orgheadline19}]{Sequent inference}
\begin{block}{Modus Ponens on Sequents}
Given the sequents 

\begin{center}
\(\Gamma \vdash P \rightarrow Q\) and \(\Delta \vdash P\), 
\end{center}

we can derive the sequent 

\begin{center}
\(\Gamma \cup \Delta \vdash Q\).
\end{center}

Challenge: The union \(\Gamma \cup \Delta\) must be computed in the derivation of this
rule.
\end{block}
\end{frame}

\begin{frame}[label={sec:orgheadline20}]{Example}
\begin{block}{Suppose we want to perform Modus Ponens on}
\begin{center}
\(P_1, P_2, P_3 \vdash P \rightarrow Q\) and \(P_1, P_3, P_4 \vdash P\)
\end{center}

where \(P_i < P_j\) for \(i,j \in \{1,2,3,4\}\).
\end{block}

\begin{block}{That is, on:}
\begin{center}
\((3, P_1 \rightarrow P_2 \rightarrow P_3 \rightarrow (P \rightarrow Q))\)
\end{center}
and 
\begin{center}
\((3,P_1 \rightarrow P_3 \rightarrow P_4 \rightarrow P)\).
\end{center}
\end{block}

\begin{block}{Goal:}
\begin{center}
\((4,P_1 \rightarrow P_2 \rightarrow P_3 \rightarrow P_4 \rightarrow Q)\).
\end{center}
\end{block}
\end{frame}

\begin{frame}[label={sec:orgheadline21}]{Computation by conversion}
First, use Axiom 1 to add extra conditions on the front of both theorems.

\begin{center}
\colorbox{green}{$P_4 \rightarrow$}\(P_1 \rightarrow P_2 \rightarrow P_3 \rightarrow (P \rightarrow Q)\)
\end{center}

and 

\begin{center}
\colorbox{green}{$P_2 \rightarrow$}\(P_1 \rightarrow P_3 \rightarrow P_4 \rightarrow P\)
\end{center}
\end{frame}

\begin{frame}[label={sec:orgheadline22}]{Computation by conversion}
Using 

\begin{center}
\((P \rightarrow Q \rightarrow R) \leftrightarrow (Q \rightarrow P \rightarrow R)\)
\end{center}

we have

\begin{align*}
                  &\colorbox{green}{$P_4$} \rightarrow P_1 \rightarrow P_2 \rightarrow P_3 \rightarrow (P \rightarrow Q)\\
  \leftrightarrow &P_1 \rightarrow \colorbox{green}{$P_4$} \rightarrow P_2 \rightarrow P_3 \rightarrow (P \rightarrow Q)\\
  \leftrightarrow &P_1 \rightarrow P_2 \rightarrow \colorbox{green}{$P_4$} \rightarrow P_3 \rightarrow (P \rightarrow Q)\\
  \leftrightarrow &P_1 \rightarrow P_2 \rightarrow P_3 \rightarrow \colorbox{green}{$P_4$} \rightarrow (P \rightarrow Q)
\end{align*}

and

\begin{align*}
                  &\colorbox{green}{$P_2$} \rightarrow P_1 \rightarrow P_3 \rightarrow P_4 \rightarrow P\\
  \leftrightarrow &P_1 \rightarrow \colorbox{green}{$P_2$} \rightarrow P_3 \rightarrow P_4 \rightarrow P
\end{align*}
\end{frame}

\begin{frame}[label={sec:orgheadline23}]{Computation by conversion}
\begin{itemize}
\item The two sequents now have the same list of assumptions in the same order.
\item We cannot apply modus ponens directly, but Axiom 2 looks promising:
\item \((\phi \rightarrow \psi \rightarrow \chi) \rightarrow (\phi
       \rightarrow \psi) \rightarrow \phi \rightarrow \chi\)
\item Let's see why:
\item (\(\phi\) \(\rightarrow\) \colorbox{red}{$\psi \rightarrow \chi$}) \(\rightarrow\) (\(\phi\)
     \(\rightarrow\) \colorbox{red}{$\psi$}) \(\rightarrow\) \(\phi\) \(\rightarrow\)
    \colorbox{red}{$\chi$}
\item We just need to collapse our cascade of implications. We can do that by rewriting
to a conjunction.
\end{itemize}
\end{frame}

\begin{frame}[label={sec:orgheadline24}]{Computation by conversion}
\small
  Using

\begin{center}
\((P \rightarrow Q \rightarrow R) \leftrightarrow (P \wedge Q \rightarrow R)\)
\end{center}

we have

\begin{align*}
                  &P_1 \rightarrow P_2 \rightarrow P_3 \rightarrow P_4 \rightarrow (P \rightarrow Q)\\
  \leftrightarrow &P_1 \wedge P_2 \rightarrow P_3 \rightarrow P_4 \rightarrow (P \rightarrow Q)\\
  \leftrightarrow &P_1 \wedge P_2 \wedge P_3 \rightarrow P_4 \rightarrow (P \rightarrow Q)\\
  \leftrightarrow &P_1 \wedge P_2 \wedge P_3 \wedge P_4 \rightarrow (P \rightarrow Q)
\end{align*}

and

\begin{align*}
                  &P_1 \rightarrow P_2 \rightarrow P_3 \rightarrow P_4 \rightarrow P\\
  \leftrightarrow &P_1 \wedge P_2 \rightarrow P_3 \rightarrow P_4 \rightarrow P\\
  \leftrightarrow &P_1 \wedge P_2 \wedge P_3 \rightarrow P_4 \rightarrow P\\
  \leftrightarrow &P_1 \wedge P_2 \wedge P_3 \wedge P_4 \rightarrow P\\
\end{align*}
\end{frame}

\begin{frame}[label={sec:orgheadline25}]{Computation by conversion}
Using axiom 2 and modus ponens, we can then obtain

\begin{center}
\(P_1 \wedge P_2 \wedge P_3 \wedge P_4 \rightarrow R\)
\end{center}

Then using 

\begin{center}
\((P \rightarrow Q \rightarrow R) \leftrightarrow (P \wedge Q \rightarrow R)\)
\end{center}

we have

\begin{align*}
                  &P_1 \wedge P_2 \wedge P_3 \wedge P_4 \rightarrow R\\
  \leftrightarrow &P_1 \wedge P_2 \wedge P_3 \rightarrow P_4 \rightarrow R\\
  \leftrightarrow &P_1 \wedge P_2 \rightarrow P_3 \rightarrow P_4 \rightarrow R\\
  \leftrightarrow &P_1 \rightarrow P_2 \rightarrow P_3 \rightarrow P_4 \rightarrow R
\end{align*}
\end{frame}

\begin{frame}[fragile,label={sec:orgheadline26}]{Conversions}
 \begin{itemize}
\item A conversion is any function which sends a term \(\phi\) to a list of theorems of the
form \(\vdash \phi \leftrightarrow \psi\).

\item The most basic conversions come from equivalence theorems:
\begin{itemize}
\item Given a theorem of the form \(\vdash \phi \leftrightarrow \psi\), we have a
conversion which:
\begin{itemize}
\item accepts a term \(t\)
\item tries to match \(t\) against \(\phi\) to give an instantiation \(\theta\)
\item returns \(\vdash \phi[\theta] \leftrightarrow \psi[\theta]\).
\end{itemize}
\item For example:
\begin{itemize}
\item the theorem \(p \leftrightarrow p\) yields a conversion called \texttt{allC}
\item the theorem \((x \leftrightarrow y) \leftrightarrow (y \leftrightarrow x)\)
      yields a conversion called \texttt{symC}
\item the theorem \((P \rightarrow Q \rightarrow R) \leftrightarrow (P \wedge Q
      \rightarrow R)\) yields a conversion called \texttt{uncurryC}
\end{itemize}
\end{itemize}
\end{itemize}
\end{frame}

\begin{frame}[fragile,label={sec:orgheadline27}]{Conversionals}
 \begin{itemize}
\item Functions which map conversions to conversions are called \emph{conversionals}.
\item Examples include:
\begin{description}
\item[{\texttt{antC}}] converts only the left hand side of an implication
\item[{\texttt{conclC}}] converts only the right hand side of an implication
\item[{\texttt{negC}}] converts only the body of a negation
\item[{\texttt{orElseC}}] tries a conversion and, if it fails, tries another
\item[{\texttt{thenC}}] applies one conversion, and then a second to the results
\item[{\texttt{sumC}}] tries all conversions and accumulates their results
\end{description}

\item With these conversionals, we can algebraically construct more and more powerful
conversions, implementing our own strategies for converting a term, such as those
we need for embedding sequent calculus.
\end{itemize}
\end{frame}

\begin{frame}[label={sec:orgheadline28}]{Truth Table Verification informally}
\begin{itemize}
\item We nominate a fresh proposition variable \(X\) and define 
\(\top \equiv X \rightarrow X\).
\item Given a proposition, we recurse on the number of other variables.
\item Base case: the only variable is \(X\). Evaluate the term according to truth table
definitions for each connective. If we evaluate to \(\top\), we have a tautology.
\item Recursive case: there are \(n\) variables other than \(X\). Take the first variable
\(P\) and consider the two cases \(P = \top\) and \(P = \bot\). Substitute in these
cases and verify that we have a tautology. If so, the original proposition is a
tautology.
\end{itemize}
\end{frame}

\begin{frame}[label={sec:orgheadline29}]{Truth Table Verification for our Sequent Calculus}
\begin{itemize}
\item Derive a rule for case-splitting:
\end{itemize}

\begin{prooftree}
  \AxiomC{$\Gamma\cup\{P\}\vdash A$}
  \AxiomC{$\Delta\cup\{\neg P\}\vdash A$}
  \BinaryInfC{$\Gamma \cup \Delta\vdash A$}
\end{prooftree}

\begin{itemize}
\item Derive theorems for evaluating tautologies:
\begin{itemize}
\item \(\top \rightarrow \top \leftrightarrow \top\)
\item \(\top \rightarrow \bot \leftrightarrow \bot\)
\item \(\bot \rightarrow \bot \leftrightarrow \top\)
\item \(\bot \rightarrow \bot \leftrightarrow \top\)
\item \(\neg\top \leftrightarrow \bot\)
\item \(\neg\bot \leftrightarrow \top\)
\end{itemize}

\item Derive
\(P \vdash P \leftrightarrow \top\) and \(\neg P \vdash P \leftrightarrow \bot\)
\end{itemize}
\end{frame}

\begin{frame}[fragile,label={sec:orgheadline30}]{Truth Table Verification for our Sequent Calculus}
 \begin{itemize}
\item Derive a conversion for fully traversing a proposition:
\end{itemize}
\begin{minted}[]{haskell}
depthC :: Conv a -> Conv a
depthC c = tryC (antC (depthC c))
           `thenC` tryC (conclC (depthC c))
           `thenC` tryC (notC (depthC c))
           `thenC` tryC c
\end{minted}

\begin{itemize}
\item Use the conversion and our evaluation rules to fully evaluate a proposition with no
variables other than \(X\). If we end up at \(\top\), we can then use the derived rule
\end{itemize}

\begin{prooftree}
  \AxiomC{$\Gamma\vdash P = \top$}
  \UnaryInfC{$\Gamma \vdash P$}
\end{prooftree}

\begin{itemize}
\item Wrap up in a verifier (and so claim our axioms complete):
\end{itemize}

\begin{minted}[]{haskell}
tautology :: Term a -> Maybe (Theorem a)
\end{minted}
\end{frame}

\begin{frame}[label={sec:orgheadline31}]{Summary}
\begin{itemize}
\item In LCF, we use a host language (ML, Haskell, Coq etc\ldots{}) to secure and program
against a trusted core.
\item A bootstrapping phase is usually required to get to the meat.
\item We can often follow textbook mathematical logic here, but we do have to worry
about computational efficiency.
\item We can embed richer logics inside the host logic (e.g. a proof tree DSL or a
sequent calculus)
\item Combinator languages can be used to craft strategies (for conversion, solving
goals with tactics)
\item With conversions at hand, problems can be converted to a form where we can
implement decision procedures and other automated tools for proving theorems
(resolution proof, linear arithmetic, computation of Grobner bases etc\ldots{})
\end{itemize}
\end{frame}
\end{document}
